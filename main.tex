\documentclass[letterpaper,12pt]{turabian-researchpaper}

%\usepackage[style=authoryear-comp,backend=bibtex]{biblatex}
\usepackage[authordate,backend=biber]{biblatex-chicago}
\usepackage[pass]{geometry}
\usepackage{palatino}
\usepackage{csquotes}
\usepackage{fancyhdr}
\usepackage{tocloft}
\usepackage{ellipsis}
\usepackage{threeparttable}

\addbibresource{dressed_to_feel.bib}

%\bibliography{abject_affect}

\begin{document}

\title{Dressed to Feel}
\subtitle{Abject Affect and the Circulation of Violence}
%\subtitle{On Vulcans, Men, and Cinema}
\author{Alexandra Chace}
\date{3 December 2018}
\course{Anthropology of Self \& Emotion}
\maketitle

\section{Introduction}

Brian De Palma's \textit{Dressed to Kill} begins in the bedroom: the camera slowly peers around the wall and into the bathroom. There, Kate Miller gazes, over her shoulder and through the shower door, to a man at the sink, shaving. Chest exposed. Kate caresses her curves as each drop flows down her body. As she gazes, we, too, gaze. Her fingers drop from breasts to bush as she looks on, the razor's motions clearing the cream from his face. She moans as her hand slides over her vulva.

\noindent Stop. 

\noindent Music shifts. She gasps. A man is behind her. He covers her mouth, forcing his fingers into her. Her body rises from the shower floor as he thrusts onto her, raping her. She screams.

\noindent Cut. 

\noindent An alarm sounds. It's the morning news. Kate's husband is on top of her, thrusting away as he always does. She moans and curls \textit{for him}. He finishes, gives her a peck, and it's off to the shower: Kate lays there, unsatisfied, unfulfilled. Wanting. Frustrated. The "wham bang special" (her own words).

Is this a nightmare? A wholesome sexual dream gone wrong? Or the kinky fantasy of the wayward housewife craving attention? Or some combination thereof? Desire is, first and foremost, mercurial. Pleasure and terror are twins, not cousins. Bedfellows. Fellow travelers. In embodiment they are easily confused---consider a scream: we scream when frightened, we scream when happy, and we scream when fucked. To pleasure and terror we respond identically: our bodies tense, adrenaline rushes through our veins. We are alert, focused, sensitive. How are these affections different?

In \textit{Unnatural Emotions: Everyday Sentiments on a Micronesian Atoll \& Their Challenge to Western Theory}, anthropologist Catherine Lutz (\citeyear{lutz_unnatural_1988}) speaks to a Western discourse of emotion that concerns itself "with emotion as essence; whether the passions are portrayed as aspects of a divinely inspired human nature or as genetically encoded biological fact, they remain, to varying degrees, things that have an inherent and unchanging nature" \autocite[53]{lutz_unnatural_1988}. In the West, we posit an "inner truth" to emotion that must "come out" through specific, universally understood actions. In this view, "...emotion is seen as a component of individuals rather than of social situations or relationships, the discipline and methods of psychology have been taken as most appropriate for its study" \autocite[41]{lutz_unnatural_1988}. \citeauthor{lutz_unnatural_1988} (\citeyear{lutz_unnatural_1988}), in response, argues for a broader interpretation of emotions as produced and implicated within certain sociocultural systems.

Kate arrives at her psychiatrist's office (Dr. Elliot), where the two discuss her lackluster sex life. She asks him if he finds her sexually attractive, to which he says yes, but stresses that it is inappropriate because she is married and is his patient. Dissatisfied, Kate leaves for a museum, where she sits, suggestively, next to a handsome, dark-haired stranger. He stands up suddenly and walks away, pretending to make note of the pieces adorning the museum walls. Kate drops her glove, pursuing the man through the museum. There is no dialogue: the music builds tension. The tension becomes terror, and Kate's face shows her fear: will Kate find him? De Palma pays careful homage to Hitchcock's \textit{Psycho}, except now the conditions reverse: she is not being chased, she is chasing. She bursts outside the building, combing the street for the dark-haired man. She spots him in a taxi, hurries into the cab, and the two strangers fuck in the backseat.

The next morning, Kate awakens in his apartment. She rummages around the room, eventually finding documents that suggest that Warren (the stranger) had contracted a sexually transmitted disease. Her pleasure at her sexual encounter quickly turns to disgust: the tiniest twist turns pleasure into terror. Disgusted and horrified, she flees his apartment and heads toward the elevator. At the elevator, she is greeted by a tall blonde woman in black sunglasses and a black trench coat. The woman reveals a razor blade, and (with comical effects) ruthlessly murders Kate before the elevator's doors close.

\textit{Dressed to Kill} stars Angie Dickinson (Kate Miller) opposite Michael Kaine (Dr. Elliot). The opening scenes I transcribe here seem, at first, of little important to Dr. Elliot's story that follows. Within 30 minutes, De Palma introduces, develops, and ultimately murders his main character and barely mentions her again.  But this is not just a convenient plot device to inaugurate Elliot's story, nor is it merely a Hitchcockian citation. Rather, \textit{Dressed to Kill}'s opening scenes establish, as a baseline, a critique of Western discourses of emotion, the very same later echoed by Lutz (1988). 

Within Western discourse, there is an assumption of difference---that there are different 'genres' of emotion entirely, each with distinct causes and physical responses, reflecting a distinct inner psychological state. With regard such difference, Derrida considered the law of genre: \textit{Genres will not be mixed. I will not mix genres}. But, as Derrida explains, this is an imposition onto the text, for nevertheless genres \textit{are} mixed. We \textit{will} mix genres. Genres are already mixed. Lutz (1998) saw the limits: we assume that emotions are discrete, oppositional, singular, but, as I will argue, this is not necessarily true. Derrida notes:
\begin{quotation}
\noindent As soon as the word "genre" is sounded, as soon as it is heard, as soon as one attempts to conceive it, a limit is drawn.  And when a limit is established, norms and interdictions are not far behind: "Do," "Do not" says "genre," the word "genre," the figure, the voice, or the law of genre. And this can be said of genre in all genres, be it a question of a generic or a general determination of what one calls "nature" or physis (for example, a biological genre in the sense of gender, or the human genre, a genre of all that is in general), or be it a question of  a typology designated as nonnatural and depending on laws or orders which were once held to be opposed to physis according to those values associated with technk, thesis, nomos (for example, an artistic, poetic, or literary genre) \autocite[56]{derrida_law_1980}.
\end{quotation}
\noindent \textit{Dressed to Kill} illustrates the mercuriality of emotion: from pleasure to horror, from horror to pleasure, initiated by minute contextual shifts. The museum scene replicates a slasher film chase perfectly: the camera is positioned such that the audience is intimately aware of her growing stress, the mood builds to frantic, she makes haste through seeming endless museum corridors, as if being chased. But Kate is not being chased---is she actually scared? How would we know?

My wager: that discourses of emotion that speak of the particular ("happiness," "sadness"; "pleasure," "terror") are prescriptive impositions, rather than natural categories. Emotion is not just socially constructed, but linguistically contingent. I argue that \textit{I am happy} and {I am sad} are not necessarily contradictory, but rather that these emotions coexist, and in fact the imposition of difference requires coexistence. The nature of emotion is instead rhizomatic---branching, connecting, multiplying, and overlapping. 

I will, through \textit{Dressed to Kill}, consider the issue of gender violence in its affective processes: \textit{Why do men hurt women they love?} Did they hate them all along? In cases of such violence, is there really a "change" or "shift" in affective sensibilities? What social and psychological purpose does violence against the feminine serve?

\section{Provisional Emotions}

We often assume that emotion, as well as identity, reflects an inner reality that is then "expressed" through actions and words. But feeling does not occur in a vacuum, and feelings are subject to cultural construction and cultural interpretation: a change in tune or angle radically changes how we feel and how others perceive our feelings. Emotion then, like identity, is an intersubjective event, rather than a subjective fact. To this regard, philosopher Judith Butler (2004a) introduces the idea of "provisionality". In her view, "[t]o claim that this is what I am is to suggest a provisional totalization of this "I"" \autocite[309]{butler_bodily_2004}. Identities produce a "radical concealment" when we attempt to "disclose the true and full content of that "I"" \autocite[309]{butler_bodily_2004}. With regard to her own lesbian identity, Butler writes:
\begin{quotation}
\noindent ...it is always finally unclear what is meant by invoking the lesbian-signifier, since its signification is always to some degree out of one's control, but also because its specificity can only be demarcated by exclusions that return to disrupt its claim to coherence. What, if anything, can lesbians be said to share? \autocite[309]{butler_bodily_2004}.
\end{quotation}

\noindent Butler's "radical concealment" is not confined to identity ,rather, we also inaugurate a "radical concealment" when we "come out" with our emotions. To say "I am happy" precludes anger, fear, or sadness. Butler rightly questions "[w]hat or who is it that is "out," made manifest and fully disclosed, when and if I reveal myself as lesbian? What is it that is now known, anything?" \autocite[309]{butler_bodily_2004}. It is impossible to account for what we preclude and what others preclude from us. Assertive or angry women, already the "emotional gender," run the risk of dismissal as irrational or hysterical. Anger can be used to discredit testimony or to incite laughter: the intent of the author is denied or channeled through other means.

As Lutz (1998) suggests, emotion "exists in a system of power relations and plays a role in maintaining it" \autocite[54]{lutz_unnatural_1988}. Emotional performance is mediated and translated by intersubjective relations of power. As such, "emotion occupies an important place in Western gender ideologies" because the "ideological subordination of women" is realized "in identifying emotion primarily with irrationality, subjectivity, the chaotic, and other negative characteristics, and in subsequently labeling women the emotional gender" \autocite[54]{lutz_unnatural_1988}. Clearly, emotions are not as clear-cut as Western theory would have us believe. Instead, emotion as a social institution presupposes certain affiliations and alliances of power.

\section{Abjection}

For Sara Ahmed, emotions "do things, and they align individuals with communities---or bodily space with social space---through the very intensity of their attachments" \autocite[119]{ahmed_affective_2004}. But emotions likewise serve to repudiate or reject others. In \textit{Dressed to Kill}, the central conflict is derived from Dr. Elliot's refusal to give his "transsexual" patient, Bobbi, approval to pursue Sex Reassignment Surgery. In response, Bobbi plots her revenge, communicating her plans with Dr. Elliot through messages left on his telephone. We eventually learn that the tall blonde that murdered Kate Miller was, in fact, Bobbi, who stole Dr. Elliot's razor from his desk. Bobbi explains:
\begin{quotation} 
\noindent This is Bobbi. You won't see me anymore, so I thought I'd have a little session with your machine. Oh Doctor, I'm so unhappy. I'm a woman trapped inside a man's body---and you're not helping me to get out! So I got a new shrink, Levy's his name, he's gonna sign the papers so I can get my operation. Oh... I borrowed your razor... and - well, you'll read all about it. Some blonde bitch saw me, but I'll get her \autocite{de_palma_dressed_2015}.
\end{quotation}
\noindent But this conflict is not what it seems: Bobbi and Dr. Elliot are actually the same person, and the conflict De Palma offers is, instead, an internal conflict between Dr. Elliot's "male side" and "female side". Dr. Levy explains:
\begin{quotation}
\noindent He was a transsexual...about to make the last step but his male side couldn't let him do it. There was Dr. Elliot, and there was Bobbi. Bobbi came to me to get psychiatric approval for a sex reassignment operation. I thought he was unstable, and Elliot confirmed my diagnosis. Opposite sexes inhabiting the same body. [The] sex-change operation was to resolve the conflict, but as much as Bobbi tried to get it, Elliot blocked it. So Bobbi got even \autocite{de_palma_dressed_2015}.
\end{quotation}
\noindent I would like to suspend discussion of the cissexist and transmisogynistic implications of this narrative. For now, I will take the film at its word for the conflict it presents. Following French philosopher Kristeva (1980) and Butler (2004b), we know that being a "man" presupposes an abjection of the motherly body and associated femininities, characterized by "an "expulsion" followed by a "repulsion" that founds and consolidates culturally hegemonic identities along sex/race/sexuality axes of differentiation" (Butler 2004, 108). The abject femininity is "expelled from the body, discharged as excrement, literally rendered "Other"" (Butler 2004, 107). To declare oneself 'man' renders femininity invisible, impossible, and elsewhere. It is important to note, however, that the expulsion is actually the process wherein "the alien is effectively established", though it "appears as an expulsion of alien elements"\autocite[107]{butler_melancholy_2004}. The establishment of the me and not-me "establishes the boundaries of the body which are also the first contours of the subject" \autocite[107]{butler_melancholy_2004}.

In Lacanian terms, young infants become self-aware through a "mirror stage", in which they, as if looking through mirrors, slowly separating themselves from their environment, eventually realizing that they are looking at their own reflections. In doing so, they form preliminary assumptions about themselves and their position within the world. \autocite[3]{lacan_ecrits_2002}. Lacan (2002) regards "the function of the mirror-stage as a particular case of the function of the imago, which is to establish a relationship between the organism and its reality" \autocite[3]{lacan_ecrits_2002}. In other words, the child makes (and is encouraged to make) certain judgment calls about what is "inner" and "outer" of the self. Lacan (2002) argues:
\begin{quotation}
	\noindent We have only to understand the mirror stage as an identification, in the full sense that analysis gives to the term: namely, the transformation that takes place in the subject when he assumes an image --- whose predestination to this phase-effect is sufficiently indicated by the use, in analytic theory, of the ancient term imago \autocite[2--3]{lacan_ecrits_2002}.
\end{quotation}
\noindent 
The issue with Freud and Lacan's "stages" discourse is that abjection is not a stage in the sense that it is bound to any given time or duration. Rather, abjection is a "stage" insofar as it provides a conceptual locale for a set of repetitive performative acts and repudiations. Once borders are created (whether national or psychological), they must be defended. \textit{Genres will not be mixed}. \textit{I will not mix genres}. \textit{Borders will not be crossed}. \textit{I will not cross borders}. Borders are imposing, they are not imposed: the alien that appears as outer is, in fact, inner. In this view, the mimetic processes produce merely the illusion of separation, rather than a true division. The alien is expelled, but never quite fully "out".

Affect is, in fact, central to the inauguration of the ego, and creates and structures borders. In her work on melancholy, Butler argues that it is "the unfinished process of grieving" that is "central to the formation of the identifications that form the ego" \autocite[245]{butler_melancholy_2004}. For her, "[t]he lost object is incorporated and phantasmically preserved in and as the ego. The lost object is incorporated and phantasmically preserved in and as the ego" (p. 245). In other words, "[m]elancholic identification permits the loss of the object in the external world because it provides a way to \textit{preserve} the object as part of the ego and, hence, to avert the loss as a complete loss" \autocite[246]{butler_melancholy_2004}. Butler (2004b) explains that "gender is acquired at least in part through the repudiation of homosexual attachments", therefore "the girl becomes a girl through being subject to a prohibition which bars the mother as an object of desire". This prohibition "installs that barred object as a part of the ego...as a melancholic identification" \autocite[248]{butler_melancholy_2004}.

But affect itself requires a series of prohibitions.. To be happy, for example, requires a repudiation of melancholy. Happiness is tautological: the antonym of "happy" is "sad". To say "I am happy" means, therefore, that "I am not sad". In order for happiness to be maintained, one must reject sadness in all its forms. We learn appropriate affect largely by mimicry and in so doing we learn which emotions must be rejected or evaded in a given context. Those who present with contextually-inappropriate emotions incite discomfort and disgust from others. Persons who are "unstable" present first and foremost with rapidly shifting and concurrent emotions. To be "stable", one must know how one feels, in discrete terms, and this feeling must be consistent through time and space. Elliot/Bobbi is unstable in part because of their failure to bring about affective cohesion by giving the impression of expulsion.

Allow me to consider an extreme example from fiction: \textit{the Vulcans}. In \textit{Star Trek}, the Vulcan's logic provides a careful foil of Western conceptions of emotion in their rejection of emotion entirely. Although Vulcans are a humanoid species, sharing many anatomical and cultural similarities with humans, among Vulcans emotion is taboo, and to express emotion or to hint at any particular affection is offensive. Instead, Vulcans prize 'logic' above all else. The most devout followers of Vulcan philosophy commit themselves to \textit{kolinahr}, a ritualistic purging of all emotion in the name of logical enlightenment \autocite{dawson_andorian_2001}.

But Vulcan logic, of course, embeds certain political judgements about emotion and its purpose. And surely, Vulcans are nevertheless \textit{affected}, despite their nominal rejection of emotion. In reality, Vulcans are \textit{repulsed} by their own emotions, and they, rather than having "successfully" purged them, work tirelessly to keep them under control. These emotions are, for the Vulcans, as taboo as the feminine or the homosexual: they are dangerous, filthy, and irrational. Kristeva (\citeyear{kristeva_powers_1982}) explains:
\begin{quotation}
	\noindent Defilement is what is jettisoned from the "symbolic system." It is what escapes that social rationality, that logical order on which a social aggregate is based, which then becomes differentiated from a temporary agglomeration of individuals and, in short, constitutes a classification system or a structure \autocite[65]{kristeva_powers_1982}.
\end{quotation}
\noindent We are who we are insofar as we embrace and repudiate certain affections. The Vulcan ideal is an impossibility: to be 'logical' does not mean unaffected. To be unaffected is impossible. As Sara Ahmed explains, "emotions are not simply "within" or "without" but that they create the very effect of the surfaces or boundaries of bodies and worlds" \autocite[117]{ahmed_affective_2004}. Similarly, \citeauthor{richard_economies_2009} (\citeyear{richard_economies_2009}) argue that affect is "mutually constitutive", meaning that subjects obtain cultural intelligibility through the many affective mimetic processes in which they are engaged. In Butlerian terms, the mimetic processes of emotionality stabilize identity (and therefore emotion) as mutually intelligible categories of analysis. Vulcans becomes legible \textit{as Vulcan} in their defilement of these emotional attachments, and by becoming \textit{Vulcan} they simultaneously refute humanity, thereby literally rendering themselves alien to the audience. It is not that Vulcans are unique in their repudiated emotions, but rather, the object of their defilement is predicated on a different symbolic system. Humans, too, require a series of repudiations to maintain legibility \textit{as human} or as a specific identity (in Dr. Elliot's case, that of a heterosexual man).

\section{Circulations of Violence}

I am aware that I have primarily discussed the individual processes of abjection and repudiation (albeit in the social context), and that I have, at times, given the impression of universalism in  working through psychoanalytic theory. But psychoanalytic theory is concerned with relations, not merely individuals. As Kristeva suggests, "abjection is coextensive with social and symbolic order, on the individual as well as on the collective level" \autocite[68]{kristeva_powers_1982}. Kristeva continues:
\begin{quotation}
\noindent By virtue of this, abjection, just like prohibition of incest, is a universal phenomenon; one encounters it as soon as the symbolic and/or social dimension of man is constituted, and this throughout the course of civilization. But abjection assumes specific shapes and different codings according to the various "symbolic systems" \autocite[68]{kristeva_powers_1982}.
\end{quotation}
\noindent We have seen a few examples of these shapes and codings in different contexts. The Vulcans provide an excellent visual exploration of the tensions that these logics inaugurate: the expulsion of internal emotions and thoughts that can never fully be removed. But De Palma provides perhaps the best example of how these seemingly internal processes become enacted onto the external world. Kristeva, similarly, argues that "internal perceptions of emotional and thought processes can be projected outwards in the same way as sense perceptions; they are thus employed for building up the external world, though they should by rights remain part of the internal world" \autocite[60]{kristeva_powers_1982}. By showing Dr. Elliot's conflict as external, rather than internal, De Palma shows us the social consequences of Dr. Elliot's own repudiation of the feminine: violence.

In this model, Elliot repudiates the feminine through outward expressions of heterosexual desire. Dr. Levy explains that "Elliot's penis became erect and Bobbi took control trying to kill anyone that made Elliot masculinely sexual" \autocite{de_palma_dressed_2015}. Because heterosexuality repudiates the feminine, his femininity lives on in his reinforced heterosexual/male identifications, and "the desire for the feminine is marked by that repudiation" \autocite[248]{butler_melancholy_2004}. Dr. Elliot "wants the woman he would never be" \autocite[248]{butler_melancholy_2004}. 

Elliot's positive feelings toward the women who arouse him are identical and simultaneous to the negative feelings he develops for Bobbi, whom he must control, defile, and excrete into the alien. Transversely, the imposition of the feminine requires control, defilement, and excretion of the masculine. For Bobbi, to murder the objects of Elliot's heterosexual attraction is to affectively foreclose on homosexual affections, for Bobbi can only be a woman insofar as she repudiates the masculine and prohibits homosexual attachments. Butler (2004b) explains:
\begin{quotation}
\noindent ...consider guilt as the turning back into the ego of homosexual attachment. If the loss becomes a renewed sense of conflict, and if the aggression that follows from that loss cannot be articulated or externalized, then it rebounds upon the ego itself, in the form of a super-ego \autocite[251]{butler_melancholy_2004}.
\end{quotation}
It is precisely \textit{because} he loves a woman that Elliot must kill a woman---himself. The "non-constitution of the (outside) object as such renders unstable the ego's identity, which could not be precisely established without having been differentiated from an other, from its object" \autocite[62]{kristeva_powers_1982}.

But this instability is not just instability of the individual, but instability of the social group: I am not speaking of Dr. Elliot, but rather of men as a social group. It is not by accident or coincidence that men are subject to intense gender policing, and that trans women, for example, who refuse masculine and heterosexual identifications, are subject to social scrutiny and intensified gender violence. Violence against women, therefore, cannot be reduced to the individual, because the disdain and rejection of the feminine and homosexual is circulatory. Ahmed (2004) explains:
\begin{quotation}
	\noindent ...hate cannot be found in one figure, but works to create the very outline of different figures or objects of hate, a creation that crucially aligns the figures together and constitutes them as a "common" threat. Importantly, then, hate does not reside in a given subject or object. Hate is economic; it circulates between signifiers in relationships of difference and displacement \autocite[119]{ahmed_affective_2004}.
\end{quotation}
For men, the origin of this hate derived from the rejection of the motherly body. The motherly body is constructed \textit{as threat}, and its rejection is internalized in and as the super-ego. From there, this rejection is displaced onto the Other---women. Men are men to the extent that they do not want another man, but men are men also in their outward displays of (trans)misogyny, which serve to strengthen their heterosexual careers. By embracing certain sensibilities and affections, we align ourselves with broader groups. In the case of men, this affect---hate---functions as rhetoric: it binds men together and restates their masculine and heterosexual identities. But hate does not precede love, for love (for something, someone) is already implicated in the imposition of hate. It cannot rightly be said that hate and love are distinct processes, they arrive simultaneously: a kind of love-hate.
%elliot is dressed to feel a certain way: his outward identity presupposes certain affections towards x or y

\section{Conclusion}

I began with a contradiction: \textit{Why do men hurt women they love?} But as I have argued, when we acknowledge the mercurial nature of affect, and the psychosocial conditions which give birth to it, there is no contradiction. In fact, there is no shift whatsoever: hate is central to the heterosexual paradigm. We cannot understand gender violence outside of this fact, for in the heteronormative arrangement, love and hate are intimate bedfellows---they constitute each other.

Mainstream narratives of gender violence, especially of the domestic variety (e.g. "that's not love, that's hate"), are useful insofar as they enable victims to come to consciousness with regard to their own experience as victims. But as many victims of such abuse will tell you, abusive partners are not \textit{always} hateful or violent, and in fact, this is a source of tension and hesitation when victims contemplate leaving abusive situations. More often than not, abusers can be quite charming, even loving. In my view, there are two possible frames. In the first, the perpetrators of gender violence have in fact \textit{always} been hateful, and that who they are in the midst of violence is who they are, who they have been, and who they will be. But this is somewhat destructive, and provides little recourse insofar as we may advocate change, because the implication is that "haters" cannot change or be reformed. In the second, we ditch the pretense of a totalizing hate, accepting that, though hate inaugurates gender violence, violence is not an absence of love or positive affection, but merely the exercising of negative affections. This frame, I think, is superior, because it allows for more nuanced understandings of violence that do not rely erroneously on the individualization of hate, but rather implicate that hate within a broader range of social phenomena.

Though our emotional signs (happy, sad, love, hate) provide useful buckets in which to categorize and situate our own experiences and those of others, they fail to adequately account for the complications of our feelings. Far from reflective of inner (stable) psychological truths, we can conceive of affect "not so much as an object circulating among subjects, but rather as a medium in which subjects circulate" \autocite[59]{richard_economies_2009}. In the model I propose, feelings are not solid objects, rather, they are liquid. Affect is rhizomorphous, which, in its Deleuzian sense, recognizes the production of "stems and filaments that seem to be roots" \autocite[15]{deleuze_thousand_1987}. Culturally imposed categories are therefore poor in their account of affect, because affect is instead a turbulent ocean, within which borders are crossed, genres are mixed, and stems are unrooted. As it is, there are no clear-cut affections in the plural, only \textit{affect} in its branching, connecting, multiplying, and overlapping implication.

%They stain us: the affective ramifications transcend the event, the stain never quite come out.

\newpage
\printbibliography

\end{document}
